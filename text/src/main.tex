%%%%%%%%%%%%%%%%%%%%%%%%%%%%%%%%%%%%%%%%%%%%%%%%%%%%%%
% Bachelor's Thesis. Nikita Mortuzaiev, FIT CVUT, 2022
%%%%%%%%%%%%%%%%%%%%%%%%%%%%%%%%%%%%%%%%%%%%%%%%%%%%%%

\documentclass[english,bachelor,unicode]{ctufit-thesis}

%%%%%%%%%%%%%%%%%%%%%%%%%%%%%%%%
% BASIC INFORMATION
%%%%%%%%%%%%%%%%%%%%%%%%%%%%%%%%
\ctufittitle{Computational Auditory Scene Analysis (CASA) for Separating Monophonic Music}
\ctufitauthorfull{Nikita Mortuzaiev} % full name including academic degrees
\ctufitauthorsurnames{Mortuzaiev} % surname(s) / family name(s)
\ctufitauthorgivennames{Nikita} % first name(s) / given name(s)
\ctufitsupervisor{Ing.\,Mgr.\,Ladislava Smítková Janků,\,Ph.D.}  % name of supervisor/advisor including academic degrees
\ctufitdepartment{Department of Applied Mathematics} % department of defence
\ctufityear{2022} % year of defence
\ctufitdeclarationplace{Prague} % place, where the declaration was signed
\ctufitdeclarationdate{May 12, 2022} % date of signature of the declaration

%%%%%%%%%%%%%%%%%%%%%%%%%%%%%%%%
% ABSTRACT AND KEYWORDS
%%%%%%%%%%%%%%%%%%%%%%%%%%%%%%%%
\ctufitabstractENG{A major challenge for modern machine hearing systems is the problem of sound source separation. One of the approaches to solve it lies in the studies in computational auditory scene analy\-sis (CASA), in which computational models are inspired by mechanisms in the human auditory system and the widely-known "cocktail party" effect. This thesis is meant to be an introduction to the field of CASA. In the first part, it investigates the theory behind auditory modeling and sound processing by diving into the physics of sound, biology of the human ear, psychology of hearing and digital signal processing. In the second part, this knowledge is practically used to implement a simple CASA system for separating monophonic music from background noise. This system is then extensively described and experimentally evaluated on a set of piano recordings.}

\ctufitabstractCZE{Velkou výzvou pro moderní systémy strojového slyšení je problém oddělení zdrojů. Jeden z přístupů k jeho řešení spočívá ve studiích počítačové analýzy sluchové scény (CASA), ve kterých jsou výpočetní modely inspirovány mechanismy sluchového ústrojí člověka a široce známým „cocktail party“ efektem. Tato práce by měla být úvodem do oblasti CASA. V první části zkoumá teorii za sluchovým modelováním a zpracováním zvuku a odkazuje se na fyziku zvuku, biologii lidského ucha, psychologii sluchového vnímání a digitální zpracování signálu. Ve druhé části jsou tyto znalosti prakticky použity k implementaci jednoduchého CASA systému pro oddělení monofonní hudby od šumu. Tento systém je pak rozsáhle popsán a experimentálně vyhodnocen na sadě klavírních nahrávek.}

\ctufitkeywordsENG{computational auditory scene analysis, auditory modeling, cocktail party effect, source separation, music processing}
\ctufitkeywordsCZE{výpočetní analýza sluchové scény, sluchové modelování, cocktail party efekt, oddělení zdrojů, zpracování hudby}

%%%%%%%%%%%%%%%%%%%%%%%%%%%%%%%%
% CUSTOMIZATION OF THE TEMPLATE
%%%%%%%%%%%%%%%%%%%%%%%%%%%%%%%%
\RequirePackage{iftex}[2020/03/06]
\iftutex % XeLaTeX and LuaLaTeX
\RequirePackage{ellipsis}[2020/05/22] % ellipsis workaround for XeLaTeX
\else
\RequirePackage[utf8]{inputenc}[2018/08/11] % this file encoding
\RequirePackage{lmodern}[2009/10/30] % vector flavor of Computer Modern font
\fi

% uncomment to emphasize hyperlinks
%\RequirePackage[pdfpagelayout=TwoPageRight,colorlinks=false,allcolors=decoration,pdfborder={0 0 0.1}]{hyperref}[2020-05-15]

% hides all hyperlinks
\RequirePackage[pdfpagelayout=TwoPageRight,hidelinks]{hyperref}[2020-05-15]

\RequirePackage{pdfpages}[2020/01/28]

\setcounter{secnumdepth}{4} % numbering sections; 4: subsubsection

%%%%%%%%%%%%%%%%%%%%%%%%%%%%%%%%
% CONTENTS SETTINGS
%%%%%%%%%%%%%%%%%%%%%%%%%%%%%%%%
\usepackage{dirtree}
\usepackage{lipsum,tikz}
\usepackage{csquotes}
\usepackage{subcaption}
\usepackage{amsmath}
\usepackage[style=iso-numeric]{biblatex}
\addbibresource{include/bibliography.bib}
\usepackage{listings} % typesetting of sources
% \usepackage{minted} % typesetting of sources

%theorems, definitions, etc.
\theoremstyle{plain}
\newtheorem{theorem}{Věta}
\newtheorem{lemma}[theorem]{Tvrzení}
\newtheorem{corollary}[theorem]{Důsledek}
\newtheorem{proposition}[theorem]{Návrh}
\newtheorem{definition}[theorem]{Definice}
\theoremstyle{definition}
\newtheorem{example}[theorem]{Příklad}
\theoremstyle{remark}
\newtheorem{note}[theorem]{Poznámka}
\newtheorem*{note*}{Poznámka}
\newtheorem{remark}[theorem]{Pozorování}
\newtheorem*{remark*}{Pozorování}
\numberwithin{theorem}{chapter}

%%%%%%%%%%%%%%%%%%%%%%%%%%%%%%%%
% MAIN PART
%%%%%%%%%%%%%%%%%%%%%%%%%%%%%%%%
\begin{document} 
	\frontmatter\frontmatterinit
	
	\includepdf{include/assignment.pdf}
	
	\thispagestyle{empty}\cleardoublepage\maketitle
	
	\imprintpage
	
	%%%%%%%%%%%%%%%%%%%%%%%%%%%%%%%%
	% TABLE OF CONTENTS
	%%%%%%%%%%%%%%%%%%%%%%%%%%%%%%%%
	\tableofcontents

	%%%%%%%%%%%%%%%%%%%%%%%%%%%%%%%%
	% LISTS OF FIGURES, TABLES, CODE LISTINGS
	%%%%%%%%%%%%%%%%%%%%%%%%%%%%%%%%
	\listoffigures
	%\begingroup
	%\let\clearpage\relax
	%\listoftables % list of tables
	%\lstlistoflistings % list of source code listings generated by the listings package
	%\listoflistings % list of source code listings generated by the minted package
	%\endgroup
	
	%%%%%%%%%%%%%%%%%%%%%%%%%%%%%%%%
	% ACKNOWLEDGEMENT
	%%%%%%%%%%%%%%%%%%%%%%%%%%%%%%%%
	\begin{acknowledgmentpage}
		I would like to express my sincerest gratitude to my supervisor Ing. Mgr. Ladislava Smítková Janků, Ph.D., as I am thankful for setting me on this new journey, guidance when the path was not clear, and encouragement in times of need. Without you, I probably would not have discovered the field of CASA at all.\\

		I am also very grateful to my dear family, friends and colleagues for the never-ending support during my studies, encouragement to broaden the horizons and develop, and inspiration to be curious and creative. My love for music and science comes from my upbringing.
	\end{acknowledgmentpage}
	
	%%%%%%%%%%%%%%%%%%%%%%%%%%%%%%%%
	% DECLARATION
	%%%%%%%%%%%%%%%%%%%%%%%%%%%%%%%%
	% INSTRUCTIONS
	% ENG: choose one of approved texts of the declaration. DO NOT CREATE YOUR OWN. Find the approved texts at https://courses.fit.cvut.cz/SFE/download/index.html#_documents (document Declaration for FT in English)
	% CZE/SLO: Vyberte jedno z fakultou schvalenych prohlaseni. NEVKLADEJTE VLASTNI TEXT. Schvalena prohlaseni najdete zde: https://courses.fit.cvut.cz/SZZ/dokumenty/index.html#_dokumenty (prohlášení do ZP)
	\begin{declarationpage}
		I hereby declare that the presented thesis is my own work and that I have cited all sources of information in accordance with the Guideline for adhering to ethical principles when elaborating an academic final thesis.\\
		
		I acknowledge that my thesis is subject to the rights and obligations stipulated by the Act No. 121/2000 Coll., the Copyright Act, as amended, in particular that the Czech Technical University in Prague has the right to conclude a license agreement on the utilization of this thesis as a school work under the provisions of Article 60 (1) of the Act.
	\end{declarationpage}
	
	%%%%%%%%%%%%%%%%%%%%%%%%%%%%%%%%
	% ASTRACT PAGE
	%%%%%%%%%%%%%%%%%%%%%%%%%%%%%%%%
	\printabstractpage
	
	%%%%%%%%%%%%%%%%%%%%%%%%%%%%%%%%
	% SUMMARY (will probably remove)
	%%%%%%%%%%%%%%%%%%%%%%%%%%%%%%%%
	%\begin{summarypage}
	%	\section*{Summary section}
	%	\lipsum[1][1-8]
	%	\section*{Summary section}
	%	\lipsum[2][1-6]
	%	\section*{Summary section}
	%	\lipsum[3]
	%	\section*{Summary section}
	%	\lipsum[2]
	%	\section*{Summary section}
	%	\lipsum[1][1-8] Lorem lorem lorem.
	%\end{summarypage}
	
	%%%%%%%%%%%%%%%%%%%%%%%%%%%%%%%%
	% ACRONYMS
	%%%%%%%%%%%%%%%%%%%%%%%%%%%%%%%%
	\chapter{Acronyms} % TODO - move to the end?
	\begin{tabular}{rl}
		ACF & Autocorrelation Function\\
		ANSI & American National Standards Institute\\
		ASA & Auditory Scene Analysis / American Standards Association\\
		CASA & Computational Auditory Scene Analysis\\
		CCF & Cross-Correlation Function\\
		CCCF & Cross-Channel Correlation Function\\
		DSP & Digital Signal Processing\\
		ERB & Equivalent Rectangular Bandwidth\\
		FIR & Finite Impulse Response\\
		IBM & Ideal Binary Mask\\
		IIR & Infinite Impulse Response\\
		JND & Just-Noticeable Difference\\
		MIDI & Musical Instrument Digital Interface\\
		PDF & Probability Density Function \\
		PreFEst & Predominant-$f_0$ Estimation Method\\
		RefraiD & Refrain Detection Method\\
		SACF & Summary Autocorrelation Function\\
		SNR & Signal-to-Noise Ratio\\
		T-F & Time-Frequency
	\end{tabular}
	
	%%%%%%%%%%%%%%%%%%%%%%%%%%%%%%%%
	% THE CONTENTS
	%%%%%%%%%%%%%%%%%%%%%%%%%%%%%%%%
	\mainmatter\mainmatterinit

	\chapter{Introduction}\label{chapter:introduction}

Imagine a party. You can hear a variety of sounds: music in the background, conversations be\-tween people, noises of somebody coughing, maybe even a dog barking outside\dots{} These sounds merge into a~single stream that approaches your ears by vibrations in the air and then goes through different physical, biological and psychoacoustical processes to finally come in a~form of electrical impulses to the brain. Despite all these sounds from different sources are mixed on the way to your ears, the brain can segregate one (or several) of them. You can focus your hearing on these “target” sounds and separate them from the complex mixture, leaving other sounds in the background. This phenomenon has been described as a “cocktail party effect”, and the process of integrating separate sounds into meaningful streams, or “auditory objects” -- auditory scene analysis, or ASA.\\

In machine perception --- specifically in machine hearing --- a related concept is referred~to as Computational ASA (CASA) and is tightly connected to the fields of sound recognition and digital signal processing. CASA systems indeed aim to separate sounds from mixtures, but they differ from BSS (blind source separation) systems in that they try to do this in a~way a~human ear does. Being based on and trying to combine works from different fields of science, CASA systems can bring new solutions and insights to the complex problem of signal separation.\\

The main objective for this thesis is to describe the principles and goals of CASA, existing applications and approaches. Another objective is to practically apply the theoretical knowledge by implementing a simple CASA system to separate monophonic music from noise, and experiment with it. But before all of this, since this thesis is made for an IT-oriented audience, it is needed to make a brief introduction to the underlying physics and biology.\\

The thesis is split into four main chapters. Chapter \ref{chapter:theory} is made to provide the theoretical background for computational auditory scene analysis and will gather related knowledge from dif\-fer\-ent fields of science: physics, biology and math. Physical background is needed for some basic understanding of what sound actually is. Since CASA processing often works with harmonic sounds (like music or voiced speech), special attention in this section will be given to describing them, along with what pitch is, and how it is perceived by the auditory system. Next, biological background theory will introduce the reader to the mechanisms in the human ear that CASA systems try to mimic. There, auditory scene analysis according to Bregman will be described as well. The following section dedicated to the related math will make an introduction to the field of digital signal processing, including a conversation about digital filters and filterbanks. And finally, background theory for computational auditory scene analysis will follow, giving a description of its main principles, goals and applications, along with an overview of three selected works in the field.\\

Then, to make a step closer to the actual implementation, a chapter dedicated to the methodology will follow. Chapter \ref{chapter:methodology} will firstly provide an overview of the mathematical concepts used in the practical part, and then will give a detailed description of the typical architecture of a CASA system.\\

Next, chapter \ref{chapter:implementation} will describe the system implemented for this thesis. There, a conversation about the selected algorithms and methods will take place, along with their input parameters. Implementation of each major stage of the architecture described previously will be given a separate section.\\

Chapter \ref{chapter:experiments} will then gather the information about the experiments made to test the implemented system, along with the observed results. Among those, experiments with white noise or other prerecorded backgrounds may be found, and an overview of the tests of the implemented system in connection with a simple classifier. The final brief section there will review other parameters of the system that may be experimented with.\\

The thesis will be concluded in chapter \ref{chapter:conclusion}, where a brief summary of what was done will take place. An overview of possible improvements and future work will follow there, creating some space for further research in the field. Contents of the enclosed medium can be found in Appendix \ref{chapter:medium}.\\

	\chapter{Theoretical Part}\label{chapter:theory}

This chapter will be dedicated to an overview of the theory behind computational auditory scene analysis. Since CASA systems are, first of all, machine listening systems, the notion of sound will be firstly defined and described from the physical point of view (section \ref{chapter:physics}). For further conversations about music processing, definitions of harmonics and pitch will take place in that same section. Then, section \ref{chapter:biology} will follow with an introduction to the biological structure of the human ear, the mechanisms in which are the primary inspiration behind CASA processing. There, the parent of CASA, auditory scene analysis, will be discussed as well. Next, section~\ref{chapter:math} will introduce the reader to the field of DSP from the mathematical point of view, providing examples of common sounds, concepts and techniques. Finally, section \ref{chapter:casa} will describe CASA itself, its principles, goals and applications, and will review three existing works in the field.

%%%%%%%%%%%%%%%%%%%%%%%%%%%%%%%%%%%%%%%
% PHYSICS
%%%%%%%%%%%%%%%%%%%%%%%%%%%%%%%%%%%%%%%
\section{Physical Background}\label{chapter:physics}

Before starting to ponder about sound source separation, it is necessary to understand the basics of how sounds work in the real world. It is safe to say that many people don’t ask this question – they just make sounds or react to them, unconsciously knowing the results. The human mind has already developed a deep understanding of which sounds are produced under different circumstances – you can easily say what to expect when somebody scratches a blackboard or rings a bell. Some could say that sounds are just “pressure waves that propagate through the air”, but in reality, there is a lot of interesting and complex things beyond this definition to pay attention to. This section will introduce the reader to the underlying physics of sound and some interesting related concepts. A special focus here will be on describing harmonic sounds, which are essential to understand to be able to work with music and pitch.

\subsection{What a Sound Is}\label{section:physics_sound}

The definition of sound above, saying that it is just vibrations in the air, is hard to be called incorrect from the scientific point of view. Of course, there are improvements to be made: for example, that sound can propagate not only through the air, but through any medium that has inert mass and is “elastic”, or stiff, meaning that it will respond to forces applied to it. Making those corrections, it is also important to note that the definition above relates to sound as a physical phenomenon, but there is another definition that people use mostly in psychology and physiology, saying that the sound is a perception in the brain, or auditory sensation of the concept described above, or “an object of hearing”. It is possible to argue about the question of “What Is Sound?” for very long time, as people still haven’t come to a single definition and tend to mix the concepts \cite{Pasnau1999}, but in this thesis, the term “sound” will be used primarily in the first, physical sense, unless specified differently.\\

For better understanding of how physical sounds work, keep in mind the mass and elasticity of the air mentioned above. Overall, mass and elasticity (not only of the air, but of any medium) play a very important role in the related studies: mass-spring systems are a highly discussed topic, along with the type of oscillations they tend to have. Any object that can produce sounds may be considered a mass-spring system: a bell, a guitar string, or even air or water, which can be thought of as many small masses connected by invisible springs\dots{} This knowledge is quite staggering -- in most cases, it is hard to imagine such a system, because there could be no obvious mass, nor elasticity. Consider an example for explaining resonant cavities: why a can of soda makes that clicking sound when it is being opened? The air is the answer. When you open the can, some parts of the air near its top act as a mass, and other parts near the bottom as a spring. The pressure in the can drops, and the “spring” at the bottom tries to suck the “mass” back in, producing the expected sound \cite{Schnupp2011}.\\

Now, if you imagine the simplest of such systems, like the one on figure \ref{img:vertical_mass_spring}, you can notice that when a particular force is applied to it, it tends to oscillate in a sinusoidal manner (due to some famous laws of physics, which will not be further discussed here). In fact, this is true for all mass-spring systems: they naturally “want” to vibrate in a sinusoidal fashion with a preferred frequency, called resonance frequency. Sinusoidal vibrations will be given some more attention in section \ref{chapter:math}.\\

\begin{figure}[t]
	\centering
	\includegraphics[height=0.27\textheight]{include/vertical_mass_spring}
	\caption[A simple vertical mass-spring system]{A simple vertical mass-spring system. Taken from \url{{https://commons.wikimedia.org/}}}
	\label{img:vertical_mass_spring}
\end{figure}

When someone talks about applying forces to objects, they can probably say that an impulse is delivered. In classical mechanics, impulse is a widely used concept, but for the purposes of this thesis, it is rather important to note how objects respond to impulses. When an impulse is delivered, the object starts to vibrate at all possible frequencies, but having in mind an understanding of resonance frequencies, it is safe to say that not all applied frequencies sound the same in the end. Thus, some tones in the resulting sound tend to be louder, and others, if not completely silent, highly attenuated. This frequency selectivity is based on the object's properties: the material of which it is made, its form, mass, \dots{} In signal processing, the notion of impulse response is widely used and will be referenced once again when discussing digital filters in section \ref{chapter:math}. The above-mentioned “chosen” frequencies will be described a bit more in the next section.\pagebreak

Another essential topic to mention here is why sounds fade in time. This is again connected to the concept of mass-spring systems and the amplitudes of their vibrations. Usually, the greater these amplitudes are, the louder the resulting sound is, so if the amplitudes didn’t become smaller, we would live in constant unbearable noise. In brief, the fading is caused by the resistance of the medium, in which the sound propagates, and the manner of this propagating. It also depends on the material of which the sound source is made. If you imagine air as it was described above --- as many masses connected by invisible springs --- the mechanics of the propagation becomes clear: the sound source pushes the closest mass near it, which due to elasticity pushes its neighbors and returns to its starting location. Then its neighbors, in turn, push their neighbors and return, and so on, until these vibrations come to your ears. The air masses must be pushed again and again for the sound to spread, so it tends to lose its strength along the way, and the further from its source it travels, the smaller the amplitudes of the vibrations become.

\subsection{Harmonicity and Pitch}\label{section:physics_harmonics_pitch}

\begin{figure}[t]
	\centering
	\includegraphics[width=0.6\textwidth]{include/harmonics}
	\caption[Harmonics of a sound wave]{First seven harmonics of a sound wave (or first seven modes of vibration of a string). Taken from \url{{https://commons.wikimedia.org/}}}
	\label{img:harmonics}
\end{figure}

The conversation about how harmonics (or overtones) appear was already started in the previous section. In simple words, not all frequencies of the vibrations caused by delivering an impulse to an object keep their amplitudes for long. The ones that benefit the most from this phenomenon are harmonics, which are the periodic waves with frequencies that are positive integer multiplications of a specific frequency called fundamental (figure \ref{img:harmonics}). For example, if the fundamental frequency is 200\,Hz, the corresponding harmonics are 400\,Hz, 600\,Hz, 800\,Hz, 1\,kHz and so on. Each harmonic can be labeled with a number -- the fundamental frequency one is also called the first, so the wave with frequency of 1\,kHz from the example above would be the fifth. However, the scientific notation for harmonics might be confusing -- some authors refer to the fundamental frequency as $f_0$ (and the fifth harmonic would be $f_4$ in that case), others as $f_1$ (and $f_5$ respectively). In this thesis, fundamental frequency will be notated as $f_0$.\\

Another explanation of how harmonics work might be found in \cite{Schnupp2011}. When you pluck a guitar string, it doesn't vibrate only as a whole. The same string might be thought of as two halves, or three thirds, or even one hundred one hundreds, and that each part of it vibrates separately. So, when the string is plucked, all its harmonics are excited, and the resulting sound is not a pure tone, but a complex one. This behavior is often called "modes of vibration" and might be observed not only in strings, but also, for example, in sheets of metal.\\

The most interesting property of harmonics is that they are all periodic at their fundamental frequency. If you sum up any number of adjacent harmonics of a wave, the period of the resulting wave would be equal to the period of the fundamental. This property plays an important role in the perception of pitch and is often used for its estimation in machine hearing systems.\\

"But what is pitch exactly?" -- you may ask. To start using this term in the thesis, it is important to provide a clear definition, but in fact, there is none that is considered a standard. The two most widely used ones were given in \cite{Plack2005}. The first one was provided by the American Standards Association (ASA) in~1960 with a reference to music -- they defined pitch as \textit{"that attribute of auditory sensation in terms of which sounds may be ordered on a musical scale"} (\cite{Plack2005}, p.\,1). The second one was given by the American National Standards Institute (ANSI) in~1994 without a reference to music, saying that \textit{"Pitch [is] that attribute of auditory sensation in terms of which sounds may be ordered on a scale extending from low to high. Pitch depends primarily on the frequency content of the sound stimulus, but it also depends on the sound pressure and the waveform of the stimulus"} (\cite{Plack2005},~p.\,1). For this thesis, it is enough to consider pitch as the auditory sensation mentioned in both definitions that can be ordered on a scale.\\

It is important to note that pitch is not a physical property of sound, but perceptual. When someone says "low pitch" or "high pitch", it is not certainly clear where this "low" or "high" is -- low pitch for some people might be high for others. In the related studies of pitch perception in psychophysics, a term "just noticeable difference" (JND) is used, and there are references that humans can distinguish about 1\,400~points on the pitch scale.\\

Pitch is often associated with fundamental frequency, though it is not fully equivalent to it. Experiments have shown that for some short periodic sounds the perception of pitch might not appear at all, though it was clear that they had a fundamental frequency. On the other hand, there are reports saying that sounds with a missing fundamental (the ones made only from higher harmonics) could evoke the perception of pitch associated with the missing frequency, thus giving the illusion of what was not present in reality \cite{Schnupp2011}. Either way, pitch is a major attribute used while describing tones in Western music, as well as is loudness, duration and timbre. Pitch also plays an important role in auditory grouping, given the fact that same sound sources tend to produce sounds that are close in pitch. Auditory grouping in humans will be given more attention in section \ref{chapter:biology}.

%%%%%%%%%%%%%%%%%%%%%%%%%%%%%%%%%%%%%%%
% BIOLOGY
%%%%%%%%%%%%%%%%%%%%%%%%%%%%%%%%%%%%%%%
\section{Biological Background}\label{chapter:biology}

Sounds are for sure one of the most important sources of information in our everyday life. By listening to them, one can describe what is happening around, understand how to react to occurring situations, or even tell if a danger is approaching, and it is time to take action. It is hard to imagine human sensation without hearing, but as easy as this may sound (no pun intended), the~biology behind it is quite complicated. This section will introduce the reader to how sound as a~mechanical phenomenon is converted to sound as perception, and will provide a~basic overview of the structures in the human ear, along with the mechanical processes happening inside of them. Neurobiological processes that happen next are not included though, and considered out-of-scope. A special guest in this section is a short overview of auditory scene analysis defined by Albert Bregman in \cite{Bregman1990}. There, basic knowledge in the field will be gathered, and some of Bregman's experiments will be described.

\subsection{Outer and Middle Ear}

At the beginning, sound approaches the ear by vibrations in the air (or any other elastic medium) and enters the outer ear, which consists of the visible part (called the auricle, or the pinna) and the ear canal. The auricle is a plate of elastic cartilage attached to the surrounding parts by muscles and other tissues. The ear canal is a tube leading from the~bottom of the~auricle to the middle ear, separated from it by the eardrum (also called the tympanic membrane). The main purpose of the ear canal is to focus the sound energy gathered by the auricle on the~eardrum. It also amplifies frequencies between 3\,kHz and 12\,kHz \cite{Standring2008}.\\

\begin{figure}[t]
	\centering
	\begin{subfigure}{0.5\textwidth}
		\centering
		\includegraphics[width=\linewidth]{include/anatomy_of_the_human_ear}
		\caption{}
		\label{img:anatomy_human_ear}
	\end{subfigure}%
	\begin{subfigure}{0.5\textwidth}
		\centering
		\includegraphics[width=\linewidth]{include/cochlea_cross_section}
		\caption{}
		\label{img:cochlea_cross_section}
	\end{subfigure}
	\caption[Anatomy of the human ear]{\textbf{(a)} Anatomy of the human ear. The ossicles of the middle ear are shown in white. The inner ear is shown in purple. \textbf{(b)} Cross-section of the cochlea showing the organ of Corti and three chambers filled with cochlear fluids. Both pictures were taken from \url{{https://commons.wikimedia.org/}}}
\end{figure}

Being gathered on the eardrum, the mechanical vibrations propagate through the middle ear. Three bones (called the ossicles) are located inside of it. The malleus (also called the hammer) is connected to the eardrum and transfers the vibrations from it to the incus (the anvil). These vibrations are chaotic, but the malleus is connected to the eardrum in a linear manner, helping the ear to respond more linearly and smoothly. The incus, in turn, connects to the stapes (the stirrup). The footplate of the stapes introduces pressure waves in the inner ear, which starts with the oval window of the cochlea. The structures of the middle ear can be seen on figure~\ref{img:anatomy_human_ear}.\\

It may sound redundant to have additional structures in the ear which propagate the vibrations even further, when they could travel just one centimeter more in a way like before, in the ear canal, but in reality, the pressure of these mechanical vibrations is too small to cause waves of the same velocity in the cochlear fluids. The ossicles help to amplify the pressure of these vibrations. They are positioned to form a lever, and because the oval window is about 14~times smaller than the eardrum, the pressure gain becomes quite significant in the end -- at~least 18.1~times \cite{Standring2008}.\\

To regulate the middle ear and protect it from damage due to very loud sounds, two muscles are located inside of it: the stapedius muscle and the tensor tympani muscle. These muscles are controlled by unconscious reflexes and hold the ossicles when the vibrations become too intense. To provide ventilation and drainage of the middle ear and to equalize pressures in this isolated environment, the middle ear is connected to the back of the throat by the eustachian tube \cite{Schnupp2011}.

\subsection{Inner Ear}

The inner ear starts with the above-mentioned oval window, which is connected to the stapes of the middle ear. The oval window is a part of the cochlea –- a structure of the inner ear dedicated to hearing. Along with the cochlea, the inner ear also contains the vestibular system, which is responsible for the sense of balance and spatial orientation and uses the same kinds of fluids and cells as the cochlea does. The vestibular system will not be covered in this thesis, but the fluids and cells will be described in more detail later in the section.\\

The cochlea itself is a spiral-shaped cavity made of bony tissue, which makes about 2.75~turns around its axis and is about 3\,cm long. The core component of it is the basilar membrane, which runs along almost its entire length and separates two of the three chambers of the cochlea filled with different fluids \cite{Schnupp2011}: the tympanic duct (scala tympani) filled with perilymph, and the cochlear duct (scala media) filled with endolymph. The third chamber, the vestibular duct (scala vestibuli), is separated from the cochlear duct by the Reissner’s membrane and is filled with peri\-lymph (figure~\ref{img:cochlea_cross_section}). When the footplate of the stapes of the middle ear introduces movements to the cochlear fluids, the basilar membrane is affected too, and the endolymph in the cochlear duct moves along.\\

The most interesting property of the basilar membrane is that its stiffness and width are not constant throughout its length – the membrane is narrow and stiff at the basal end of the cochlea, and wide and floppy at the apical end. And here sound waves have two possible routes to take while propagating through the basilar membrane: a shorter path, which includes going through the stiffer parts of it, or a longer path, which means travelling along the membrane until it becomes easier to pass through, but pushing more fluid on the way. In fact, high-frequency waves tend to choose the shorter path, and low-frequency waves – the longer one. The distribution of frequencies passing through the basilar membrane is not linear, but close to logarithmic. In machine hearing systems, equivalent rectangular bandwidth (ERB) scale is usually used to estimate it (see chapter \ref{chapter:methodology} for a definition).\\

So, the basilar membrane moves in different places depending on the frequencies of the vibrations. The organ of Corti, which sits on top of it and runs along its entire length, contains displacement cells able to detect movements of the fluid nearby and excite the nearby neurons to send electrical impulses. Such cells are packed with a bunch of stereocilia (hair) that stick out of its top, and thus are called hair cells. These cells can be of two types: inner hair cells that are located closer to the center of the cochlea, and outer hair cells that sit closer to its outer side. Inner hair cells are less numerous than outer hair cells and form a single row along the organ of Corti, while outer hair cells usually form three rows \cite{Schnupp2011}.\\

Now, it is important to mention that the endolymph in the cochlear duct contains high amounts of positively charged ions (primarily potassium and calcium). When it moves in response to the sound pressure, the stereocilia of the inner hair cells are deflected, and tiny ion channels open in them. This allows the charged ions from the endolymph to enter the stereocilia. The cell becomes depolarized, and a receptor potential is produced. This results in releasing the neurotransmitters at the basal end of the cell and then triggering action potentials in the nerve nearby. In this way, inner hair cells detect movements around them, and mechanical sound waves are converted to electrical nerve signals.\\

Outer hair cells, in turn, serve as amplifiers of quiet sounds. Their receptor potentials are converted to cell body movements, and as a result, the sound pressure increases \cite{Hudspeth2008}.

\subsection{Auditory Scene Analysis}\label{section:biology_asa}

To close up the biological background section, it was decided to make an introduction to auditory scene analysis according to Bregman \cite{Bregman1990}. His book named \textit{"Auditory Scene Analysis: The Perceptual Organization of Sound"} (1990) made a big influence on further research, as it attempted to bring together theoretical knowledge in the field that did not have any clear base to build on. Bregman's book is now widely recognized as this base, so it is necessary to address at least the primary concepts of ASA described there. This section could have been put to either of the chapters in the theoretical part of the thesis, because it is connected to every field being discussed, but it resides in the biological part, because most of the addressed experiments were testing human auditory perception and are highly connected to the related studies in Gestalt psychology.\\

To start off, Bregman brings to the world a new term related to auditory perception. If you recall the definitions of sound from section \ref{chapter:physics}, you may remember that there were two of them: one related to sound as a physical phenomenon, and another related to perception in the brain. Bregman introduced a term "auditory stream", or "auditory object", to address the second definition. He made an analogy with vision and how humans tend to group separate surfaces of the same object on their eye retina to see the object as a whole and referred to "auditory streams" as to the same kind of objects, but for audition. He said that the term "sound" is not really well suitable in this case, because for example a melody in a recording of music consists of different sounds (notes), but people often percept this melody as a whole and group the sounds into something greater in their perception. Bregman's definition of auditory streams became very popular, so it will be used throughout this thesis too.\\

Bregman defines ASA as the process of separating these auditory streams from mixtures and refers to it as a two-stage process. The first stage (segmentation) is said to be splitting the auditory input into so-called "segments", just as a visible object is split into surfaces in the human eye. The second stage is grouping and refers to integrating the segments together based on the grouping cues. With references to experiments from his lab, he describes two possible approaches to grouping and searching for cues: simultaneous (which is also called vertical, or spectral) and sequential (or horizontal). While simultaneous grouping takes into account the segments that appear at the same time, but relate to different frequencies (are spread in space), sequential grouping works with segments that share the frequency component, but are located at different points in time. As an example of a cue for simultaneous grouping one could take common onset and offset, because it is usual for sounds (or different frequency components of the same sound) from the same source to start at the same time. For sequential grouping, pro\-bab\-ly the most common cue is pitch. If two sounds have pitches associated with fundamental frequencies that are close to each other, the sounds will likely be grouped into the same stream. When the frequencies are further away, the sounds will most certainly appear in different streams.\\

For a demonstration of sequential grouping, consider a galloping sound that consists of two alternating tones: "A-B-A". The pattern repeats endlessly with a velocity set by the researcher. The fundamental frequencies of the tones A and B can also be set beforehand. Experimentally \cite{Schnupp2011}, it was discovered that when the fundamental frequencies are close to each other and/or the speed of repeating the pattern is small, then there is only one resulting stream in the perception of the sequence. On the other hand, when the frequencies are further away from each other and/or the sequence is being repeated at a faster rate, the subjects report that they can hear two streams: one consisting of repeating A-sounds ("A-{}-{}-A-A-{}-{}-A"), and another of repeating B-sounds ("-{}-B-{}-{}-{}-{}-B-{}-"). In this case, the subjects also reported that they could focus their hearing only on one of the two above-mentioned streams, and the other one was rather heard in the background. Interestingly, when the repeating rate and the difference in frequencies were set to some specific values in between, the subjects reported that the perception of two streams was alternating with the perception of one stream every 15-20~seconds. This phenomenon was experimentally verified during the research for this thesis with the help of the website made for \cite{Schnupp2011}.\\

In his book \cite{Bregman1990}, Bregman described a similar experiment, but the repeating pattern was more complex. There were six tones: three lower ones (1,~2~and~3) and three higher ones (4,~5~and~6), and they were repeating in a pattern like this: "1-4-2-5-3-6". Bregman asked the subjects to report the order of the heard tones, and at faster rates of repeating they were failing to do this for both groups at the same time. When they were focusing their hearing on the lower tones, the higher ones were heard in the background, and it was difficult for them to correctly determine the order.\\

As an example for simultaneous grouping, Bregman makes another experiment. This time, the pattern consists of two alternating sounds: a pure tone (A) and a complex tone that consists of two pure tones (B and C). Again, the pattern repeats endlessly, and both the speed of repetition and fundamental frequencies of the tones might be changed. As a result, Bregman reports that the tone B was \textit{"an object of rivalry"} (\cite{Bregman1990}, p.654): in cases when A was close to it in frequency, they were grouped together in a simultaneous manner, but when they were further away from each other, the tone B was rather grouped simultaneously with C, creating a richer tone BC.\\

Bregman's theory is highly related to the studies in Gestalt psychology. In his book, he was drawing parallels between vision and hearing, and found a lot of similarities between them that were supported by the Gestalt laws of grouping. He described the concepts of "belongingness" and "exclusive allocation", and the principles of similarity, proximity and closure. Also, he was questioning whether scene analysis is an innate process, or the one acquired by learning.\\

Finally, to make a parallel with computer modeling, Bregman referred to the notion of heuristics. In his words, heuristics are \textit{"the procedures that are not guaranteed to solve the problem, but are likely to lead to a good solution"} (\cite{Bregman1990}, p.\,32). He believed that outputs from multiple heuristics should be used at the same time to find the good solution, and that there are similar mechanisms in human perception. For example, when there is evidence about common onsets and offsets of two frequency components of a sound, and it was supported by the fact that these components were different harmonics of the same fundamental frequency, the probability of being incorrect after guessing that these components should be grouped in the same stream becomes very close to zero. Here, two heuristics contributed to the decision of whether to group the sounds into a single stream.

%%%%%%%%%%%%%%%%%%%%%%%%%%%%%%%%%%%%%%%
% MATH
%%%%%%%%%%%%%%%%%%%%%%%%%%%%%%%%%%%%%%%
\section{Mathematical Background}\label{chapter:math}

Next, to address the practical part of the thesis, it is needed to give some attention to the underlying math. Considering that the system described in chapter \ref{chapter:implementation} extensively uses techniques and concepts from the field of digital signal processing, it was decided to make a brief introduction to the basics of it for a reader that might be confused by the variety of new terms. DSP, however, is a very large discipline, so the overview in this section should not be considered fully exhaustive. Having said all this, the present section will firstly describe how sounds are represented computationally, and will provide a few examples of the most common ones. Then, some common techniques and concepts from the field of DSP will be described, including the notion of a digital filter and a filterbank. Other specific mathematical concepts used in the implementation part will be addressed later in chapter \ref{chapter:methodology}.

\subsection{The Basics of Digital Signal Processing}\label{section:math_basics}

As it was described in section \ref{chapter:physics}, physical sounds in real world spread in the environment in a form of pressure waves. These waves are continuous, so to be able to work with sounds via computers it is usually useful to convert them to some kind of a discrete representation. The notion of a discrete, or discrete-time, signal is used in these cases and is defined as a time series sampled at equally-spaced points on the time axis, or as a function of discrete time ($x(n)$, for example) \cite{Shenoi2005}. The discrete signal's sampling frequency $f_s$ is defined as a number of samples observed during a unit of time. Digital signals, in turn, are encoded representations of discrete-time signals. Sometimes, they may also be represented as vectors \cite{Abood2020}: \begin{equation}
	\textbf{x} = [x(0), x(1), \dots{}, x(N - 1)]^T\qquad\textbf{x}\in\mathbb{R}^N
\end{equation}
where $N$ is the overall number of samples.\\

If you recall the conversation about resonant frequencies and the sinusoidal manner of vibrations from section \ref{chapter:physics}, you may remember that when an impulse is delivered to an object, the object responds to it by entering into vibrations. It starts to vibrate at all possible frequencies, but not all of them survive. An important mathematical instrument for frequency analysis is the Fourier transform (along with the Fourier series). The Fourier theorem states that any periodic function might be represented as a sum of sines and cosines, so technically, any waveform (including the ones for sound waves) might be decomposed and represented in such a way. This decomposition may also be given by an amplitude spectrum and a phase spectrum \cite{Schnupp2011}.\\

The most basic example of such frequency decomposition is for a pure tone (the first row on figure \ref{img:windowing_example}). Pure tones are impossible to find in nature, or to even perfectly produce with a speaker. Pure tones produced computationally sound flat and unnatural, but they are the basic building blocks of other sounds. The waveform of a pure tone is a sine wave and is defined as a function of time $t$:
\begin{equation}
	y(t) = a\sin(2\pi{}ft + \varphi)
\end{equation}
where $f$ is the frequency, $a$ is the amplitude and $\varphi$ is the phase. The frequency decomposition of a pure tone will contain only one peak at frequency $f$. A pure tone and its frequency decomposition might be seen on the first row of figure \ref{img:windowing_example}.\\

Another example of a common sound would be a click. Clicks are instant modulations in amplitude of a sound waveform, or waves that instantly go up and down at certain points in time. The most interesting fact about a click is its frequency decomposition being an infinite set of sine waves. More about clicks can be found, for example, in \cite{Schnupp2011}.\\

The last important sound that will be mentioned here is white noise. The waveform of the white noise is completely random, so its frequency decomposition is random too. The white noise is used in the thesis for experiments with the implemented CASA system and its ability to separate music.\\

\begin{figure}[t]
	\centering
	\includegraphics[width=0.75\textwidth]{include/windowing_example}
	\caption[An example of windowing and the problem of discontinuities]{The problem of discontinuities that arises when rectangular windows are used. The first row depicts a 1\,kHz pure tone (a sine wave) and its amplitude spectrum. The second row demonstrates the amplitude spectrum of the same tone masked by a rectangular window. The third row shows the spectrum of the same tone masked by a Hanning window. The window functions are shown in gray. Taken from \cite{Schnupp2011}.}
	\label{img:windowing_example}
\end{figure}

Continuing the conversation about the frequency decomposition, it is necessary to note that it is impossible to extract any information about the time component from the output of the Fourier transform. Thus, it is useful to firstly split the wave into separate intervals on the time axis (which are often called windows), and only then compute their frequency decompositions. The resulting time-frequency representation of the sound wave is called a spectrogram. An example of a spectrogram is shown on figure \ref{img:spectrogram_example}.\\

However, there is a known problem that emerges when windowing is used with the Fourier transform. When the window is rectangular (the wave is cut off vertically from both sides) and is not aligned with the period of the signal, the onset and offset of the wave become abrupt. These sudden changes in amplitude result in the necessity of adding countless additional sine waves to the frequency decomposition, so it begins to contain chaotic information, which makes the valuable parts of the spectrum less precise. An example of this behavior was given in \cite{Schnupp2011} and is shown on the second row of figure~\ref{img:windowing_example}.\\

\begin{figure}[t]
	\centering
	\includegraphics[height=0.25\textheight]{include/spectrogram_example}
	\caption[An example of a spectrogram]{A spectrogram of a male voice saying "ta-ta-ta". Time is shown on the horizontal axis, and frequencies are shown on vertical. The color intensity increases with density. Three separate syllables are clearly visible. Taken from \url{{https://commons.wikimedia.org/}}}
	\label{img:spectrogram_example}
\end{figure}

A viable solution of this discontinuity problem comes with attempts to smooth the abrupt ends of the masked wave. Windows that have some kind of ramping on both sides are used in this case. The ramping helps to smoothly turn the sound on and off and reduce the "spectral splatter" \cite{Schnupp2011}. An example of such window (Hanning window) is shown on the third row of fig\-ure~\ref{img:windowing_example}.\\

It is also worth noting that when the masking window is short, the resulting amplitude spectrum becomes wider, and vice versa: when precise frequency representation is needed, the time window must be wide enough. This property is called time-frequency trade-off and can be observed in spectrograms: the spectrograms with high frequency resolution usually have low time resolution, and the ones with high time resolution have low frequency resolution.

\subsection{Filters and Filterbanks}\label{section:math_filters}

In section \ref{chapter:physics}, when there was a conversation about the objects' impulse responses, some attention was given to the selectivity of frequencies. It was said that frequencies that don't align with the object's resonance frequency are attenuated, or filtered out. Thus, the mentioned object might be thought of as a mechanical linear filter. The linearity comes from the fact that the force applied to the object is proportional to the amplitude of the output signal (in simple words, the harder you pluck the guitar string, the louder the resulting sound would be). The linear filter's impulse response is a function of time that depicts how it responds to a simple external impulse, and its frequency response is a function that shows how much different frequencies are affected after the filtering.\\

The filters used in the implementation part of the thesis are digital, meaning that they ope\-rate on discrete-time or digital signals by performing different mathematical operations. Their counterpart is analog filters operating on continuous-time (also called analog) signals \cite{Shenoi2005}. Linear filters might be of two types: infinite impulse response (IIR) or finite impulse response (FIR). Impulse response of an IIR filter does not become equal to zero after a certain point in time, but continues infinitely, whereas impulse response of an FIR filter is given only for a certain time interval. FIR filters are usually non-recursive and less efficient, while IIR filters are recursive and computationally better.\\

The technique that is used for filtering is called convolution. For digital signals, it is defined as follows \cite{Schnupp2011}:
\begin{equation}
	(f*g)(n) = \sum_{m=0}^{N-1}f(m)g(n - m) = \sum_{m=0}^{N-1}f(n - m)g(m)
\end{equation}
where $f(n)$ is the input signal, $g(n)$ is the filter's impulse response, $m$ is the delay, or lag, and $N$ is the overall number of samples. Convolution is commutative, thus the functions for the input signal and the filter impulse response may be swapped.\\

The last term for this section is a filterbank. Basically, a filterbank is a collection of filters with different properties. In the implementation part, a filterbank of gammatone filters is used to simulate the basilar membrane of the human inner ear.

%%%%%%%%%%%%%%%%%%%%%%%%%%%%%%%%%%%%%%%
% CASA
%%%%%%%%%%%%%%%%%%%%%%%%%%%%%%%%%%%%%%%
\section{Computational ASA}\label{chapter:casa}

Having described the underlying concepts from different fields of science in previous sections, it is time to finally focus on computational auditory scene analysis. CASA is said to be the study that groups practical, programmable solutions for auditory scene analysis problems, or the study of ASA by computational means. CASA systems are used primarily for source se\-pa\-ra\-tion, meaning that they are machine listening systems that aim to separate "target" sounds from mixtures, just like people do when they try to focus on a specific "stream" and not to be distracted by others. In that CASA systems differ from systems for blind signal separation – they try to mimic (at least to some extent) the mechanisms inside the human ear discussed in section~\ref{chapter:biology}, or are inspired by various experiments in neurobiology and psychoacoustics \cite{VanDerKouwe2001}. This section will be an attempt to gather different authors' opinions and works in the field, and will conclude the theoretical part of the thesis. In its first part, main principles of CASA and its goals and applications will be addressed, while the second part will be dedicated to an overview of three selected existing works.

\subsection{Principles, Goals and Applications}

To start off, it was hard for the author to define multiple principles that the researched CASA systems have in common. The most prominent one is the number of microphones used to record the input sounds, being equal to one (in monaural models) or two (in binaural ones) -- to simulate one or two human ears. Monaural models are researched better, but can't be used for extracting features like location of the sound source, which is possible to some extent in binaural models, where time differences between sound energy levels in the two recordings might be used for computing it.\\

And that is pretty much all that was found in common across all systems -- even their architecture is not the same, and is always adapted to the specific problem being solved (the one discussed in chapter \ref{chapter:methodology} is used primarily for speech segregation by certain authors). If one speaks about music processing, multiple techniques in the context of CASA were reviewed by Goto in \cite{Wang2006}. There, the problem of "music scene description" was addressed, which focused on describing the "scenes" that occur in a musical performance (such as melody and bass lines, beat structure, or chorus sections), and building an understanding of them, rather than fully separating the sound sources. In this context, even Bregman in his newer paper pointed out that \textit{"there is evidence that the human brain does not completely separate sounds"} (\cite{Bregman1995}, p.\,134).\\

That is, of course, important to keep in mind when setting goals for the CASA system being implemented. Most authors, however, still develop successful systems for source separation that in the end \textit{"\dots{}might be dealing with a problem which is not solved by any mechanism in this world, not even by the human brain, although sound source segregation is valuable from the viewpoint of engineering"} (\cite{Goto2004}, \cite{Wang2006}, p.\,251).\\

Thus, looking at all of this from that "engineering viewpoint", most authors make references to the goals of ASA, which, according to the Bregman's book discussed in section \ref{chapter:biology}~\cite{Bregman1990}, has to produce separate streams from the auditory input. For CASA, Wang and colleagues \cite{Wang2005} proposed that the goal should be to find an ideal binary mask (IBM) for the time-frequency (T-F) representation of the input. If the input is split into T-F units, where time is on the horizontal axis and frequency on the vertical, the IBM is a binary matrix that has ones in places where the target sound prevails, and zeroes elsewhere for background units. Like that, the input sound mixture might be split into streams computationally, and the connection to ASA is kept. Ideal binary masks (and time-frequency masks overall) will be given some more attention in chapter \ref{chapter:methodology}.\\

The research of CASA systems and their applications in science \cite{Szabo2016} have been quite diverse recently. As it was already mentioned, some of the models are inspired by various biological experiments \cite{Wang2008}\cite{Boes2011}, while others are trying to address the cocktail party problem in natural environments \cite{Elhilali2008}. Some models try to explicitly simulate perceptual data, but others may refer to perception only very slightly. The expected output for the system implemented in this thesis is to find an IBM to be able to mask different backgrounds, when target sounds are monophonic piano music.\\

Aside from pure scientific interest, CASA systems find useful applications in everyday life \cite{Wang2006}. Some of them are listed below.

\begin{description}
	\item[Automatic speech recognition] Apparently, speech recognition is the most popular field, where CASA systems have been used. Many speech recognition systems have performance losses in acoustic environments, where multiple sources of sound are present. The development is often put in contrast with computer vision systems that basically fulfill the same purpose, but for another human sense.
	\item[Automatic music transcription] A complex problem on its own (even human experts can come up with different solutions) becomes more complicated when multiple musical instruments are involved and need to be transcribed separately. CASA can certainly bring new insights to the field.
	\item[Hearing prostheses] Modern hearing aids made for people suffering from hearing loss don't separate speech in noisy environments, amplifying the background too. CASA systems may become a good front-end for this problem, considering their ability to filter the noise out.
	\item[Audio information retrieval] Recordings on the Internet usually contain mixtures of sounds from different sources, and it is necessary to separate (or at least describe) them to be able to search efficiently. By employing techniques for music scene description, "music thumbnails" (the most representative sections, mainly chorus) can be computed to simplify the searching.
\end{description}

\subsection{Selected Works}

In this section, three selected works from the field of CASA will be reviewed. Since this thesis aims to process musical signals, the above-mentioned problem of music scene description will be addressed firstly, and in this context two techniques will be presented. Then, a system that gave an inspiration for the one implemented in the thesis (however made to segregate speech) will be given some attention as well.\\

The first work that the author would like to have described in the thesis is research done by Masataka Goto in \cite{Goto2004} in 2004. There, the problem of music scene description was described firstly, followed by a proposition of a model able to estimate melody and bass lines in recordings of musical performances, called PreFEst (predominant-$f_0$ estimation method).\\

To start off, Goto assumed that human listeners understand the properties of the sounds that they hear in mixtures, and thus are able to build descriptions of distinct auditory objects in them. He said that \textit{"This understanding, however, is not necessarily evidence that the human auditory system extracts the individual audio signal corresponding to each auditory stream"} \cite{Goto2004} and that segregation is not necessarily a condition for understanding. That is, we don't usually need to completely separate sources from mixtures to be able to say that there are multiple of them. Goto addressed CASA systems as the ones aiming to build an essential description of the auditory scene, and defined the problem of auditory scene description. Music scene description is, in turn, such description of musical scenes.\\

With that said, Goto proposed a real-time method called PreFEst that was able to detect melody and bass lines in monaural sound mixtures. In contrast to the earlier models, PreFEst did not predefine the number of sound sources, did not locally trace frequency components and did not assume the existence of the fundamental frequency component \cite{Goto2004}\cite{Wang2006}. The model consisted of three parts: PreFEst-front-end, PreFEst-core and PreFEst-back-end. The front-end part took care of the frequency analysis stage, and was splitting the frequency components into two sets: one for the melody line (medium and high frequencies) and one for the bass line (low frequencies). After that, each set of components was represented as a probability density function, called an observed PDF, and was used in the PreFEst-core part to estimate the PDF of the $f_0$. Finally, the back-end part of the model was determining the temporal continuity of the estimated $f_0$, and forming the resulting melody and bass lines (the trajectories of their movements). What needs to be noted here is that the estimated lines were not split into distinct note sequences, but were rather descriptions of them, and therefore needed further computations to be converted, for example, into MIDI scores.\\

The next model for music scene description was proposed by Goto in \cite{Goto2006} in 2006. Called \mbox{RefraiD} (refrain detection method), the model was able to detect chorus or other repeating sections in popular-music recordings. With chorus sections being \textit{"the most representative, uplifting, and prominent thematic sections in the music structure of a song"} (\cite{Goto2006}, \cite{Wang2006}, p.\,275), the proposed system enabled a possibility to create so-called "thumbnails" for songs, which can be used to either find desired songs, or advertise them to customers, for example. The model was novel in that it was addressing different problems that might have appeared while searching for chorus sections -- they didn't necessarily need to be exactly the same, but, for example, written in a different key (or modulated). With that said, the model was firstly extracting "chroma-vectors" that consisted of 12 elements (one for each pitch class -- C, C\#, D, \dots{}, B), then calculated similarities between them, listed and integrated the repeated sections (including the modulated ones), and, finally, using the "chorus measure" selected the chorus sections from other repeated ones.\\

Overall, Masataka Goto's research in music scene description has been a big inspiration for the author, however, since it was discovered quite late, it is, unfortunately, reflected here very briefly, and is not used in any way in the practical part of the thesis.\\

The model that did become the inspiration for the practical part, is the one implemented by Guoning Hu and DeLiang Wang in 2010 -- the tandem algorithm for pitch estimation and voiced speech seg\-re\-ga\-tion~\cite{Hu2010}. There, the authors referred to the interchangeably dependent problems of voice separation and pitch estimation as to a "chicken and egg" question, because better pitch estimates would be produced from a clearer signal, but at the same time source separation would give better results, if there are good pitch estimates for the target sounds. To address this problem, the authors used an iterative approach to gradually refine outputs from both stages, and that is where the name comes from -- the two techniques were used in tandem. Besides that, the first two preparation stages of the system architecture were similar to the ones that will be described in chapter \ref{chapter:methodology} -- the authors firstly created a T-F representation of the input signal by computing a cochleagram (using 128 gammatone filters with center frequencies uniformly distributed on the ERB-rate scale), and then, for the feature extraction stage, computed a correlogram, summary autocorrelation and cross-channel correlation for both input sound wave and its envelope. These functions (except for the envelope function) will be defined in chapter \ref{chapter:methodology}, and then used to describe the implemented system architecture in chapters \ref{chapter:methodology} and \ref{chapter:implementation}.\\

For the tandem algorithm, the main two stages were implemented as follows. At the voice separation stage, the main goal of which was to find an IBM given the target pitch, the authors used multi-layer perceptrons to predict whether a T-F unit should be masked or not, based on either its individual properties, or the information from its T-F neighborhood. Then, to estimate pitch given the target mask, they tested two different approaches -- one using the autocorrelation function and the other using probabilities of distinct T-F units being target-dominant. Some other techniques like temporal continuity were used next to refine these estimates.

% TODO: Other approaches to source separation (+comparison): Beamforming/ICA?

\section{Chapter Summary}

This chapter focused on a discussion about the background theory behind computational auditory scene analysis. Theoretical knowledge from different fields (physics, biology, ASA, DSP and CASA) was gathered here, and a stable base for further research was provided. Some valuable works were also reviewed here, including Albert Bregman's book about auditory scene analysis \cite{Bregman1990}, Masataka Goto's research in music scene description \cite{Goto2004}\cite{Goto2006}, and Guoning Hu and DeLiang Wang's tandem algorithm for pitch estimation and voiced speech segregation \cite{Hu2010}. The next chapter will be dedicated to a closer overview of the mathematical concepts, methods and algorithms used in the implementation part of the thesis.

	\chapter{Methodology}

\section{Mathematical Concepts Used in the Thesis}\label{section:math_concepts}

This section will provide more specific information about the mathematical concepts used in the implementation part. They are listed below.

\begin{description}
	\item[Equivalent rectangular bandwidth] Equivalent rectangular bandwidth, or ERB, is a measure used for computing bandwidths of the filters for human auditory system. It was defined by Moore and Glasberg in 1983 as \cite{Moore1983}\cite{Holdsworth1988}:
	\begin{equation}
		ERB(f) = 6.23f^2 + 93.39f + 28.52
	\end{equation} 
	
	In 1990, the authors published another approximation (linear) \cite{Moore1990}\cite{Wang2006}:
	\begin{equation}
		ERB(f) = 24.7\,(4.37f + 1)
		\label{equation:ERB_1990}
	\end{equation}
	where $f$ is frequency in kHz.
	
	\item[ERB-rate scale] In psychoacoustics, ERB-rate scale is used to uniformly distribute the filter center frequencies based on their ERB bandwidths. This scale is similar to the critical-band scale of the human auditory system. In 1983, Moore and Glasberg defined it as follows \cite{Moore1983}:
	\begin{equation}
		E(f) = 11.17\ln\left|\frac{f + 0.312}{f + 14.675}\right| + 43.0
	\end{equation}
	
	Using the latest approximation of ERB by Moore and Glasberg (1990), ERB-rate scale function is approximated as \cite{Moore1990}\cite{Wang2006}:
	\begin{equation}
		E(f) = 21.4\log_{10}\left(0.00437f + 1\right)
	\end{equation}
	
	\item[Gammatone filter] A gammatone filter is a linear filter, whose impulse response is a product of a sinusoidal tone and gamma function \cite{Wang2006}:
	\begin{equation}
		g_{f_c}(t) = at^{L-1}\,e^{-2\pi{}tb(f_c)}\,\cos(2\pi{}f_c{}t + \varphi)\,u(t)
		\label{equation:gammatone_impulse_response}
	\end{equation}
	where $a$ is the filter amplitude, $L$ is its order (number of iterations of filtering), $f_c$ is its center frequency, $\varphi$ is the phase, $u(t)$ is the unit-step function ($u(t) = 1$ for $t \ge 0$, and $0$ otherwise), and $b(f_c)$ is the function that determines the bandwidth for a given center frequency \cite{Wang2006}:
	\begin{equation}
		b(f) = 1.019\,ERB(f)
	\end{equation}
	
	Gammatone frequency response is defined as follows \cite{Holdsworth1988}:
	\begin{equation}
		G(f) = \left[1 + \frac{j\,(f - f_c)}{b(f_c)}\right]^{-L} + 
		\left[1 + \frac{j\,(f + f_c)}{b(f_c)}\right]^{-L}
		\qquad\left(-\infty < f < \infty\right)
	\end{equation}
	
	However, when modeling human auditory system, the second term from the definition above can be ignored for sufficiently large $\frac{f_c}{b(f_c)}$ \cite{Wang2006}\cite{Holdsworth1988}:
	\begin{equation}
		G(f)\approx\left[1 + \frac{j\,(f - f_c)}{b(f_c)}\right]^{-L}
		\qquad\left(0 < f < \infty\right)
	\end{equation}
	
	\item[Autocorrelation] Autocorrelation, or autocorrelation function (ACF), is a function that is used to find periodicities and other cues in the input signal. It is defined as the correlation of the signal with its shifted copy. In this thesis the simulated auditory nerve responses will be used to compute it \cite{Wang2006}, and the result will be normalized to receive a set of Pearson's coefficients:
	\begin{equation}
		ACF(n, c, \tau) = \frac{
			\sum\limits_{k=0}^{K-1}a(n - k, c)\,a(n - k -\tau, c)
		}{
			\sqrt{\strut\sum\limits_{\tau}{a^2(n - k, c)}}\,
			\sqrt{\strut\sum\limits_{\tau}{a^2(n - k - \tau, c)}}
		}\,h(k)
	\label{equation:ACF}
	\end{equation}
	where $a(n, c)$ represents the simulated auditory nerve response for frequency channel $c$ and discrete time $n$, $\tau$ is the time lag, $K$ is the length of the sampling window, and $h(k)$ is the window function (usually Hanning, exponential or rectangular).
	
	\item[Summary autocorrelation] Summary autocorrelation function, or SACF, is defined as \cite{Wang2006}\cite{Wang2012}:
	\begin{equation}
		SACF(n, \tau) = \sum_c{ACF(n, c, \tau)}
		\label{equation:SACF}
	\end{equation}
	
	\item[Cross-channel correlation] Cross-channel correlation is a correlation between signals from different frequency channels. In this thesis it is defined for each two neighboring channels ($c$ and $c + 1$) in each time window as normalized correlation \cite{Wang2006}\cite{Wang2012}:
	\begin{equation}
		CCCF(n, c) = \frac{
			\sum\limits_{k=0}^{K-1}a(n - k, c)\,a(n - k, c + 1)
		}{
			\sqrt{\strut\sum\limits_{k=0}^{K-1}{a^2(n - k, c)}}\,
			\sqrt{\strut\sum\limits_{k=0}^{K-1}{a^2(n - k, c + 1)}}
		}h(k)
		\label{equation:CCCF}
	\end{equation}
	where $a(n, c)$ is the simulated auditory nerve response for frequency channel $c$ and discrete time $n$, $K$ is the length of the sampling window, and $h(k)$ is the window function.
\end{description}

\section{Architecture of a CASA System}\label{section:casa_architecture}

Next, to successfully address a concrete implemented system, it is necessary to understand its architecture. The architecture described in the following subsections is based on \cite{Wang2006}, \cite{Wang2012} and \cite{Jasti2020}, though it is impossible to say that it is used in all systems -- in different sources the authors use different approaches and methods, and thus different structures of the models.

\subsection{Peripheral Analysis}\label{subsection:casa_peripheral_analysis}

Usually, a model for computational auditory scene analysis begins with the peripheral analysis of the input sound. Here, first preparations of the input sound for further processing take place. The expected result of this stage includes a time-frequency representation of the input sound -- a set of so-called T-F units. Since the models try to mimic the human cochlea --- and a lot of scientific attention was given to researching it --- the outcome of this stage is almost always a cochleagram.\\

Cochleagrams are in many cases produced by a filterbank of gammatone filters (see chap\-ter~\ref{section:math_concepts} for a definition). Gammatone filters were picked as the most precise ones to simulate points on the basilar membrane of the inner ear. The number of filters in the filterbank might be chosen by the researcher, but the most frequent choice is $128$. The center frequencies of the filters are distributed on the ERB-rate scale (see chapter \ref{section:math_concepts} for a definition), as it was developed to be similar with the critical-band scale of the human auditory system. The filters' impulse and frequency responses are shown on figure \ref{img:gammatone_filterbank}, however it is worth noting that the filters on the figure are normalized, and in practice frequencies higher than 2\,kHz are often amplified \cite{Wang2006}.\\

\begin{figure}[t]
	\centering
	\begin{subfigure}{0.48\textwidth}
		\centering
		\includegraphics[width=\linewidth]{include/gammatone_filterbank_impulse_response}
		\caption{}
		\label{img:gammatone_filterbank_IR}
	\end{subfigure}%
	\begin{subfigure}{0.52\textwidth}
		\centering
		\includegraphics[width=\linewidth]{include/gammatone_filterbank_frequency_response}
		\caption{}
		\label{img:gammatone_filterbank_FR}
	\end{subfigure}
	\caption[Gammatone filterbank impulse and frequency responses]{\textbf{(a)} Impulse responses of $15$ gammatone filters in a gammatone filterbank. The filters' center frequencies are equally spaced between 400\,Hz and 10\,kHz on the ERB-rate scale and are shown on the vertical axis. Taken from \cite{Schnupp2011}. \textbf{(b)} Frequency responses of the same filters. The vertical axis depicts the changes in amplitudes in decibels. Note that the filters with higher center frequencies have wider bandwidths.}
	\label{img:gammatone_filterbank}
\end{figure}

If you look at a cochleagram and a spectrogram of the same sound at the same time, you probably won't notice many differences. The cochleagram similarly shows the densities of frequencies at different points in time, though in this case it may be thought of as a set of frequency channels, where each channel is the output of a certain gammatone filter from the filterbank. The gammatone filters are meant to change the signal so that the frequencies near the center frequency are kept, and the ones that are further away are attenuated (see figure \ref{img:gammatone_filterbank_FR}). At this stage, the windowing techniques described in chapter \ref{section:math_basics} are also used for long signals.

% TODO: Meddis hair cells?

\subsection{Feature Extraction}\label{subsection:casa_feature_extraction}

After the cochleagram is computed, a classic CASA system involves computing a correlogram. In this thesis, correlograms will be put to the feature extraction stage, however some sources discuss them along with the cochleagrams in the peripheral analysis stage \cite{Jasti2020}.\\

According to \cite{Wang2006}, the term correlogram was introduced by Slaney and Lyon in 1990 \cite{Slaney1990} as \textit{"an animated picture of the sound that shows
	frequency content along the vertical axis and time structure along
	the horizontal axis"} (in this thesis the correlogram is a three-dimensional array). The authors used autocorrelation function (see chapter \ref{section:math_concepts} for a definition) on each frequency channel to compute it and described that if the sound is periodic, then the ACF will have peaks in places corresponding to the lags that are equal to the period of repetition. This property of the autocorrelation function has been extensively used in signal processing to estimate pitch and the corresponding fundamental frequencies.\\

So, the fundamental frequency is estimated using the correlogram and the representations of the higher harmonics in it. If you recall the conversation about harmonics from chapter \ref{section:physics_harmonics_pitch}, you may remember that harmonics are periodic at their fundamental frequency, so if the fundamental frequency is, for example, 100\,Hz (the period is 10\,ms), the second harmonic is 200\,Hz (5\,ms), and thus repeats every 10\,ms as well. The correlogram depicts this property in different frequency channels, and when the autocorrelations are summed up, the resulting summary ACF (see chapter \ref{section:math_concepts} for a definition) will have peaks in places corresponding to the period of the fundamental frequency.\\

Of course, correlogram might be used not only for the $f_0$-estimation. In the implementation part, for example, cross-channel correlation is computed from it too. Cross-channel correlation was defined in chapter \ref{section:math_concepts} according to \cite{Wang2012} and may be used to find similarities between units in adjacent frequency channels.\\

Some authors also work with cross-correlograms at this stage. According to \cite{Wang2006}, cross-correlogram is based on the simulated auditory nerve responses from the left and right ears (thus it may be used in binaural systems) and is defined as follows:
\begin{equation}
	CCF(n, c, \tau) = \sum_{k=0}^{K-1} a_L(n-k, c) a_R(n-k-\tau, c) h(k)
\end{equation}
where $a_L(n, c)$ and $a_R(n, c)$ is the above-mentioned simulated auditory nerve response from the left and right ears respectively, $\tau$ is the lag, $h(k)$ is the window function, and $K$ is the size of the sampling window.

\subsection{Mid-Level Representation and Scene Organization}\label{subsection:casa_segmentation_and_grouping}

Next, after the feature extraction stage, follows some work related to segmentation and grouping, which according to Bregman are the main two stages of auditory scene analysis (see chapter \ref{section:biology_asa}). For this part, different authors come up with different names and implementations, but in this thesis it will be split into two stages as in \cite{Wang2006} and \cite{Jasti2020}: mid-level representation and scene organization.\\

Mid-level representation stage includes the first stage of Bregman's ASA --- segmentation. Here, the T-F units computed during the peripheral analysis stage are segregated into segments with the help of the extracted features. These segments, or mid-level representations, are meant to split the cochleagram into separate continuous parts and then are grouped to form auditory streams.\\

Scene organization stage, as it was mentioned, serves to group the mid-level representations into meaningful auditory objects, or streams (see chapter \ref{section:biology_asa}) that come from individual sound sources. Here, for example, separate harmonics are grouped together to form the richer musical sound, or the melody played by the pianist's right hand is grouped with the accompaniment played by the left hand. As a part of this stage, various techniques from the field of artificial intelligence might be used to achieve better results.\\

A notable outcome of this stage is a time-frequency mask of the input cochleagram (or any other chosen T-F representation). The main idea behind masking is to emphasize the T-F units corresponding to the target sound and attenuate those from the background. The mask might have either binary or real values from the $[0, 1]$ interval. If the former method is chosen, the task of searching the T-F mask might be interpreted as a binary classification problem. In case of the latter representation, the values from the mask might be interpreted differently: for example, as the signal-to-noise ratios (SNR) that show the ratios or differences between the target sound energy and overall energy in the T-F unit, or as probabilities that the T-F unit belongs to the target sound \cite{Wang2006}.\\

As it was mentioned earlier, Wang and colleagues \cite{Wang2005} proposed ideal binary mask (IBM) to be the primary goal of CASA. The IBM is defined as follows \cite{Wang2006}\cite{Wang2012}:
\begin{equation}
	IBM(n, c) = 
	\begin{cases}
		1 & \quad\text{if } SNR(n, c) \ge LC \\
		0 & \quad\text{otherwise}
	\end{cases}
\end{equation}
where $SNR(n, c)$ is the signal-to-noise ratio, or the difference between the target sound energy and the interference energy in the T-F unit, and $LC$ is the local criterion chosen as a threshold for the SNR function (most commonly 0\,dB).

\subsection{Resynthesis}\label{subsection:casa_resynthesis}

Finally, some systems include the resynthesis stage to convert the masked T-F representation back to the sound waveform. This is useful for evaluation of the resulting CASA system in cases when listening experiments are held, or signal-to-noise ratios need to be compared before and after the processing \cite{Wang2006}. The technique for resynthesis was proposed by Weintraub in his PhD~thesis in 1985 \cite{Weintraub1985}: the outputs from each gammatone filter in the filterbank were reversed in time to remove the phase shifts, then passed backwards through each filter and reversed back in time again.

	\chapter{Implementation}\label{chapter:implementation}

The implementation chapter will provide specific information about the CASA system implemented for the thesis. It will give more attention to the algorithms used for different parts of the architecture and their input parameters. The main goal of the system was to separate monophonic piano music from the background noise by finding an ideal binary mask for the cochleagram. The follow-up experiments will be described later in chapter \ref{chapter:experiments}.\\

It should also be noted, that the implemented system is rather simple, thus it should not be expected to observe source separation of too high quality. The implemented model serves only as an example of a CASA system and is not aiming to separate the sources using the most modern and sophisticated algorithms and approaches.\\

The system is implemented in Python with the help of \textit{numpy} \cite{Harris2020}, \textit{scipy} \cite{Virtanen2020}, \textit{statsmodels} \cite{Seabold2010}, \textit{matplotlib} \cite{Hunter2007} and \textit{brian2hears} \cite{Stimberg2019brian2hears} packages. \textit{skimage} \cite{VanDerWalt2014} and \textit{brian2} \cite{Stimberg2019brian2} were used as supporting ones. The main Python scripts contain functions for different parts of the resulting architecture, and then an example of their usage along with the experiments overview is given in the supporting Jupyter notebooks (see the structure of the attached medium for more information).

\section{Cochleagram}

The cochleagram described in chapter \ref{subsection:casa_peripheral_analysis} was implemented with the help of \textit{brian2hears}~\cite{Stimberg2019brian2hears} package. At the beginning, an array of center frequencies was computed using the ERB-rate scale defined in chapter \ref{section:math_concepts}. In the main example notebook, there were 128 center frequencies uniformly distributed on it between the values of the lowest and the highest fundamental frequencies on the standard piano keyboard (containing 88 keys) -- from $27.5$\,Hz (note $A_0$) to $4.186$\,kHz (note $C_8$).\\

As a next step, a gammatone filterbank was used to split the input into 128 corresponding frequency channels. The filters were implemented as cascades of four IIR filters of order~2 (which corresponds to single gammatone filters of order~4 \cite{Stimberg2019brian2hears}). The approximate impulse response was similar to the one defined in equation \ref{equation:gammatone_impulse_response}:
\begin{equation}
	g_{f_c}(t) = t^3\,e^{-2\pi{}bERB(f_c)t}\,\cos(2\pi{}f_c{}t)
\end{equation}
where $b=1.019$ and the ERB function was the same as in equation \ref{equation:ERB_1990} for frequency in Hertz. Finally, a unit-step function was applied to the output along with a cubic root function that helped to better emphasize low amplitudes in the cochleagram. The resulting cochleagrams for C-major and A-major scales are shown on figure \ref{img:cochleagram_example}.\\

\begin{figure}[t]
	\centering
	\begin{subfigure}{0.5\textwidth}
		\centering
		\includegraphics[width=\linewidth]{include/cochleagram_example_C-major}
		\caption{}
	\end{subfigure}%
	\begin{subfigure}{0.5\textwidth}
		\centering
		\includegraphics[width=\linewidth]{include/cochleagram_example_A-major}
		\caption{}
	\end{subfigure}
	\caption[Comparison of cochleagrams for C-major and A-major scales]{Cochleagrams for C-major \textbf{(a)} and A-major \textbf{(b)} scales. The ascending and descending note progressions and the note harmonics are clearly visible in both cases, as well as the difference in fundamental frequency between the tones (all notes from A-major scale are higher).}
	\label{img:cochleagram_example}
\end{figure}

After the cochleagram was computed, is was needed to split the output into windows for further processing. A rectangular window of size 20\,ms was used as a default with an overlap of 10\,ms.

\section{Correlogram and Other Features}

For the feature extraction stage described in chapter \ref{subsection:casa_feature_extraction}, a correlogram was implemented using the autocorrelation function implemented in the \textit{statsmodels} package. The provided implementation could compute the ACF similarly as defined in equation \ref{equation:ACF}, however its another variant that uses Fast Fourier Transform was used for higher efficiency. The default number of lags for the autocorrelation function was chosen to be equal to the number of samples in the sampling window, i.\,e.~20\,ms times the samplerate of the input sound (48\,kHz).\\

Thus, the resulting correlogram was a three-dimensional array of floats in $[-1, 1]$ range. The first dimension was time frames, the second was frequency channels and the third was lags for the autocorrelation function.\\

Next, a summary autocorrelation function was computed to help with extracting the fundamental frequencies for separate time frames. The formula was exactly the same as in equation~\ref{equation:SACF}. Also, for demonstration purposes, cross-channel correlation was computed as defined in equation~\ref{equation:CCCF}.\\

Finally, the SACF function was used to estimate the fundamental frequencies of the signals in each time frame. For this step, the "dominant" lags were firstly found, meaning equally spaced lags with the highest sum of SACF values, and then the fundamental frequency was estimated for the current time frame using the distance between two adjacent lags. The default number of "dominant" lags was initially chosen to be 5. The resulting correlogram for time frame 150 is~shown on figure \ref{img:correlogram_example} along with all extracted features.

\begin{figure}[t]
	\centering
	\includegraphics[width=\textwidth]{include/correlogram_example}
	\caption[An example of correlogram and the extracted features for C-major scale]{An example of a correlogram for C-major scale for time frame 150. Note that distinct frequency channels have similar repeating patterns corresponding to the repetitions observed in harmonics. The pattern starts around the frequency channel 40 and repeats twice as fast around channel 60, then three times as fast around channel 70, and so on. Cross-channel correlation is shown on the left panel, and the summary autocorrelation is shown on the bottom panel. Note the peaks on the plot for SACF that emerge when all harmonics become "synchronized". Their corresponding lags are shown as green lines, and the lag corresponding to the fundamental frequency is marked as a red dot.}
	\label{img:correlogram_example}
\end{figure}

\section{Masking}




	\chapter{Experiments and Results}\label{chapter:experiments}


\section{Dataset Overview}\label{section:experiments_dataset}

The set of input data for testing of the implemented system consisted of 34 piano recordings of various scales and intervals. The files were converted to WAV file format to address the limitations of the used libraries. Below is a more detailed overview:

\begin{itemize}
	\item A-major scale played fast and slow in ascending and descending order, starting from $A_3$
	\item A-minor scale played fast and slow in ascending and descending order in three modes: natural, harmonic and melodic, starting from $A_3$
	\item C-major scale played fast and slow in ascending and descending order, starting from $C_4$
	\item C-major scale, where each note is repeated 3 times, in ascending order in four variants: starting from $C_3$, $C_4$, $C_5$ and $C_6$
	\item D-major scale played fast and slow in ascending and descending order, starting from $D_4$
	\item E-major scale played fast and slow in ascending and descending order, starting from $E_4$
	\item E-minor scale played fast and slow in ascending and descending order in three modes: natural, harmonic and melodic, starting from $E_4$
	\item F-major scale played fast and slow in ascending and descending order, starting from $F_4$
	\item G-major scale played fast and slow in ascending and descending order, starting from $G_4$
	\item H-major scale played fast and slow in ascending and descending order, starting from $H_4$
	\item Perfect melodic fourths, where the lower notes are in range starting from $C_3$ and ending at $C_6$, in ascending order
	\item Perfect melodic octaves, where the lower notes are in range starting from $C_3$ and ending at $F_5$, in ascending order
	\item All semitones (or all keys one after another), starting from $C_4$ and ending at $H_5$, in ascending order
	\item All semitones (or all keys one after another), starting from $C_2$ and ending at $H_3$, in ascending order
\end{itemize}

\section{Experiments with White Noise}

The first set of experiments involved experiments with white noise levels. The noise level was an input parameter of the CASA system an was used before the peripheral analysis stage to modify the input sound. The functionality for adding the white noise was provided by the \textit{brian2hears} \cite{Stimberg2019brian2hears} package, and the noise level parameter was used as the amplitude of the white noise sound wave, which was then added to the input sound wave. The set of values for the noise level parameter was as follows: $\left\{0;~0.005;~0.01;~0.02;~0.04\right\}$. The results are shown on figure \ref{img:white_noise_experiments}.\\

On the figure, the first row depicts the results for the input sound without artificially added white noise, and each next row shows the results for different noise levels. When comparing the cochleagrams, the rising amounts of background noise can be clearly seen. From the resulting masking, it can be observed that the model began to lose the quality of source separation when the noise level was around $0.02$, which is clearly a major amount of randomness in the input, and can be well heard by a human listener. Given the simplicity of the algorithm for the segmentation-and-grouping stage, the outcomes can be considered a success.

\begin{figure}[t]
	\centering
	\includegraphics[width=\textwidth]{include/experiments_white_noise_A-major_0.jpg}
	\includegraphics[width=\textwidth]{include/experiments_white_noise_A-major_0,005.jpg}
	\includegraphics[width=\textwidth]{include/experiments_white_noise_A-major_0,01.jpg}
	\includegraphics[width=\textwidth]{include/experiments_white_noise_A-major_0,02.jpg}
	\includegraphics[width=\textwidth]{include/experiments_white_noise_A-major_0,04.jpg}
	\caption[Results of experiments with white noise levels]{Results of the experiments with white noise levels for A-major scale played fast. In each row, the first image is a cochleagram of the input sound, the second is the estimated IBM, and the third is the masked cochleagram.}
	\label{img:white_noise_experiments}
\end{figure}

\section{Other Experiments}
	\chapter{Conclusion}

The main goal of this thesis was to describe computational auditory scene analysis. For this task, theoretical background was provided in chapter \ref{chapter:theory}, which hopefully helped the reader to dive into the topic and get used to the main concepts and ideas. There, sounds were firstly discussed from the physical point of view, and particular attention was given to harmonic sounds and the perception of pitch. Then, biological theory followed, including an overview of the structures in the human ear and an introduction to Bregman's~ASA. Next followed a section dedicated to the basics of digital signal processing, along with a brief conversation about digital filters and filterbanks. Finally, chapter \ref{chapter:theory} also described the principles, goals and applications of CASA, and reviewed major works in the field.\\

Chapter \ref{chapter:methodology} gathered and described some related mathematical concepts, such as ERB-rate scale, gammatone filter and autocorrelation function. It also included an overview of the architecture of a typical CASA system, and its main stages: peripheral analysis, feature extraction, mid-level representation, scene organization and resynthesis. There, the problem of finding an ideal binary mask for the cochleagram was referred to as the main objective of CASA.\\

Chapter \ref{chapter:implementation} addressed the next objective, and as a result, a simple CASA system was implemented to separate monophonic piano music from background noise. The main focus in this chapter was given to describing the used algorithms and their input parameters. In the end, the system appeared to give appropriate results, although the techniques used at some stages were quite primitive.\\

Chapter \ref{chapter:experiments} was the final one and provided an overview of the experiments for the implemented system. The dataset for them contained a variety of piano recordings, among which were diffe\-rent scales and intervals. They were processed either separately, or in mixtures with white noise or other backgrounds. A memory-based approach was tested as well, giving considerably better results if compared with straight-forward attempts for source separation.\\

Overall, the objectives set for the thesis were successfully achieved. For the author, this thesis became a big inspiration to continue studying digital signal processing, psychoacoustics and music. The final section will be dedicated to an overview of what can be done next.

\section{Future Work}

The field of computational auditory scene analysis is relatively new and, of course, requires more scientific attention. In the author's opinion, the main reason why it is not yet extensively researched and did not catch every scientist's eye is that it requires deep knowledge in several fields that are not usually taught in parallel. This thesis provided a decent proof of this: to begin talking about CASA it was appropriate to introduce the reader into the underlying physics and biology, as well as to digital signal processing, which is often given separate university courses. Thus, the possibilities for improvements come from different fields of science, but could bring valuable solutions to all of them at once.\\

The first improvement the author sees for the implemented system is the one for the segmen\-tation-and-grouping stage. In practice, different authors approach it differently and bring solutions that could have not many things in common, but almost all of them refer to the Bregman's definition of ASA. Thus, for a CASA system it is not very natural to work with all time-frequency units at once and compute the resulting mask by exploiting their common features. More sophisticated techniques are usually employed in this case, including, for example, machine learning algorithms for the grouping stage.\\

Another limitation was that the output masks were binary, and thus background sound energy observed in unmasked T-F units was not attenuated at all. A possible improvement in this case would be to work with real-valued masks that were briefly discussed in chapter \ref{subsection:casa_segmentation_and_grouping}.\\

The next task may be to remove the word "monophonic" from the name of the thesis. This will include research of the algorithms for multiple $f_0$ estimation and solving the related problems, one of the main ones being, for example, how to estimate the fundamental frequency of a note that is the same as one of the harmonics of the other note. A separate chapter in \cite{Wang2006} is dedicated for this task and may be taken as a starting point.\\

Of course, the above-mentioned polyphony may be applied differently. Some may refer to it in terms of a single musical instrument, when two or more notes are played simultaneously, while others may think of it as played by multiple instruments at once. Each of these cases brings new challenges to the computational models for the cocktail party problem, but they are certainly worth the work to be done.\\

Another improvement might address binaural recordings. In fact, binaural sounds enable the possibility of applying sound localization techniques, which employ the notions of interaural time and intensity differences. The approaches to sound localization might involve computing a cross-correlogram, which was given some attention in the thesis. Overall, the topic is also quite interesting and challenging, and may bring new solutions to the feature extraction stage. The chapter dedicated to binaural sound localization from \cite{Wang2006} might be studied for further inspiration.\\

Finally, if one returns back to the basic fact that CASA systems try to simulate the human ear, further improvements can be made in the "amount" of this simulation. This thesis was mimicking the activity of the basilar membrane of the inner ear by using a cochleagram, however many authors improve the outcomes by involving models that simulate neural activity of the hair cells. A notable example of such model might be found under the name "Meddis hair cell".

	
	\appendix\appendixinit
	\chapter{Contents of the Enclosed Medium}

\dirtree{%
	.1 README.md\DTcomment{installation guide and folder structure}.
	.1 requirements.txt\DTcomment{required Python packages}.
	.1 src.
	.2 scripts\DTcomment{Python source codes}.
	.2 main\_example.ipynb\DTcomment{Jupyter notebook with an example of the system usage}.
	.2 experiments.ipynb\DTcomment{Jupyter notebook with experiments}.
	.1 text.
	.2 src\DTcomment{\LaTeX{} source code for the text}.
	.2 main.pdf\DTcomment{PDF-formatted text of the thesis}.
	.1 data.
	.2 target\_sounds\DTcomment{prerecorded target monophonic piano sounds}.
	.2 background\_sounds\DTcomment{prerecorded background sounds and noises}.
	.2 masks\DTcomment{precomputed ideal binary masks for clean target sounds}.
	.2 output\DTcomment{outputs from the Jupyter notebooks}.
}

	
	\backmatter
	\printbibliography
\end{document}
