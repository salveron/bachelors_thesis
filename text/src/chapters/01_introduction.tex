\chapter{Introduction}\label{chapter:introduction}

Imagine a party. You can hear a variety of sounds: music in the background, conversations be\-tween people, noises of somebody coughing, maybe even a dog barking outside\dots{} These sounds merge into a~single stream that approaches your ears by vibrations in the air and then goes through different physical, biological and psychoacoustical processes to finally come in a~form of electrical impulses to the brain. Despite all these sounds from different sources are mixed on the way to your ears, the brain can segregate one (or several) of them. You can focus your hearing on these “target” sounds and separate them from the complex mixture, leaving other sounds in the background. This phenomenon has been described as a “cocktail party effect”, and the process of integrating separate sounds into meaningful streams, or “auditory objects” -- auditory scene analysis, or ASA.\\

In machine perception --- specifically in machine hearing --- a related concept is referred~to as Computational ASA (CASA) and is tightly connected to the fields of sound recognition and digital signal processing. CASA systems indeed aim to separate sounds from mixtures, but they differ from BSS (blind source separation) systems in that they try to do this in a~way a~human ear does. Being based on and trying to combine works from different fields of science, CASA systems can bring new solutions and insights to the complex problem of signal separation.\\

The main objective for this thesis is to describe the principles and goals of CASA, existing applications and approaches. Another objective is to practically apply the theoretical knowledge by implementing a simple CASA system to separate monophonic music from noise, and experiment with it. But before all of this, since this thesis is made for an IT-oriented audience, it is needed to make a brief introduction to the underlying physics and biology.\\

The thesis is split into four main chapters. Chapter \ref{chapter:theory} is made to provide the theoretical background for computational auditory scene analysis and will gather related knowledge from dif\-fer\-ent fields of science: physics, biology and math. Physical background is needed for some basic understanding of what sound actually is. Since CASA processing often works with harmonic sounds (like music or voiced speech), special attention in this section will be given to describing them, along with what pitch is, and how it is perceived by the auditory system. Next, biological background theory will introduce the reader to the mechanisms in the human ear that CASA systems try to mimic. There, auditory scene analysis according to Bregman will be described as well. The following section dedicated to the related math will make an introduction to the field of digital signal processing, including a conversation about digital filters and filterbanks. And finally, background theory for computational auditory scene analysis will follow, giving a description of its main principles, goals and applications, along with an overview of three selected works in the field.\\

Then, to make a step closer to the actual implementation, a chapter dedicated to the methodology will follow. Chapter \ref{chapter:methodology} will firstly provide an overview of the mathematical concepts used in the practical part, and then will give a detailed description of the typical architecture of a CASA system.\\

Next, chapter \ref{chapter:implementation} will describe the system implemented for this thesis. There, a conversation about the selected algorithms and methods will take place, along with their input parameters. Implementation of each major stage of the architecture described previously will be given a separate section.\\

Chapter \ref{chapter:experiments} will then gather the information about the experiments made to test the implemented system, along with the observed results. Among those, experiments with white noise or other prerecorded backgrounds may be found, and an overview of the tests of the implemented system in connection with a simple classifier. The final brief section there will review other parameters of the system that may be experimented with.\\

The thesis will be concluded in chapter \ref{chapter:conclusion}, where a brief summary of what was done will take place. An overview of possible improvements and future work will follow there, creating some space for further research in the field. Contents of the enclosed medium can be found in Appendix \ref{chapter:medium}.\\
