\chapter{Conclusion}

The main goal of this thesis was to describe computational auditory scene analysis. For this task, theoretical background was provided in chapter \ref{chapter:theory}, which hopefully helped the reader to dive into the topic and get used to the main concepts and ideas. There, sounds were firstly discussed from the physical point of view, and particular attention was given to harmonic sounds and the perception of pitch. Then, biological theory followed, including an overview of the structures in the human ear and an introduction to Bregman's~ASA. Next followed a section dedicated to the basics of digital signal processing, along with a brief conversation about digital filters and filterbanks. Finally, chapter \ref{chapter:theory} also described the principles, goals and applications of CASA, and reviewed major works in the field.\\

Chapter \ref{chapter:methodology} gathered and described some related mathematical concepts, such as ERB-rate scale, gammatone filter and autocorrelation function. It also included an overview of the architecture of a typical CASA system, and its main stages: peripheral analysis, feature extraction, mid-level representation, scene organization and resynthesis. There, the problem of finding an ideal binary mask for the cochleagram was referred to as the main objective of CASA.\\

Chapter \ref{chapter:implementation} addressed the next objective, and as a result, a simple CASA system was implemented to separate monophonic piano music from background noise. The main focus in this chapter was given to describing the used algorithms and their input parameters. In the end, the system appeared to give appropriate results, although the techniques used at some stages were quite primitive.\\

Chapter \ref{chapter:experiments} was the final one and provided an overview of the experiments for the implemented system. The dataset for them contained a variety of piano recordings, among which were diffe\-rent scales and intervals. They were processed either separately, or in mixtures with white noise or other backgrounds. A memory-based approach was tested as well, giving considerably better results if compared with straight-forward attempts for source separation.\\

Overall, the objectives set for the thesis were successfully achieved. For the author, this thesis became a big inspiration to continue studying digital signal processing, psychoacoustics and music. The final section will be dedicated to an overview of what can be done next.

\section{Future Work}

The field of computational auditory scene analysis is relatively new and, of course, requires more scientific attention. In the author's opinion, the main reason why it is not yet extensively researched and did not catch every scientist's eye is that it requires deep knowledge in several fields that are not usually taught in parallel. This thesis provided a decent proof of this: to begin talking about CASA it was appropriate to introduce the reader into the underlying physics and biology, as well as to digital signal processing, which is often given separate university courses. Thus, the possibilities for improvements come from different fields of science, but could bring valuable solutions to all of them at once.\\

The first improvement the author sees for the implemented system is the one for the segmen\-tation-and-grouping stage. In practice, different authors approach it differently and bring solutions that could have not many things in common, but almost all of them refer to the Bregman's definition of ASA. Thus, for a CASA system it is not very natural to work with all time-frequency units at once and compute the resulting mask by exploiting their common features. More sophisticated techniques are usually employed in this case, including, for example, machine learning algorithms for the grouping stage.\\

The next task may be to remove the word "monophonic" from the name of the thesis. This will include research of the algorithms for multiple $f_0$ estimation and solving the related problems, one of the main ones being, for example, how to estimate the fundamental frequency of a note that is the same as one of the harmonics of the other note. A separate chapter in \cite{Wang2006} is dedicated for this task and may be taken as a starting point.\\

Of course, the above-mentioned polyphony may be applied differently. Some may refer to it in terms of a single musical instrument, when two or more notes are played simultaneously, while others may think of it as played by multiple instruments at once. Each of these cases brings new challenges to the computational models for the cocktail party problem, but they are certainly worth the work to be done.\\

Another improvement might address binaural recordings. In fact, binaural sounds enable the possibility of applying sound localization techniques, which employ the notions of interaural time and intensity differences. The approaches to sound localization might involve computing a cross-correlogram, which was given some attention in the thesis. Overall, the topic is also quite interesting and challenging, and may bring new solutions to the feature extraction stage. The chapter dedicated to binaural sound localization from \cite{Wang2006} might be studied for further inspiration.\\

Finally, if one returns back to the basic fact that CASA systems try to simulate the human ear, further improvements can be made in the "amount" of this simulation. This thesis was mimicking the basilar membrane of the inner ear by implementing a cochleagram, however many authors improve the outcomes by involving models that simulate neural activity of the hair cells. A notable example of such model might be found under the name "Meddis hair cell".
