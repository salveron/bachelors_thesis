\chapter{Computational ASA}\label{chapter:casa}

Now, having described all the underlying concepts from different fields of science in previous chapters, it is time to finally focus on computational auditory scene analysis. CASA is said to be a field of study that groups practical, programmable solutions for auditory scene analysis problems, and thus can introduce new discoveries and insights to it. CASA systems are used primarily for source separation, meaning that they are machine listening systems that aim to separate sounds from different sources in mixtures. However, they are not the same as systems for blind signal separation – the core difference is that CASA systems try to mimic (at least to some extent) the mechanisms inside the human ear, which were described in chapter \ref{chapter:biology}. In this chapter, main principles of CASA systems will be described, along with a typical structure, desired outputs and applications. In the second part, major works that use computational auditory scene analysis for source separation will be reviewed and compared.

\section{Typical Structure of a CASA System}