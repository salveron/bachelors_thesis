\chapter{Physical Background}

Before starting to ponder the structures of the human ear, it is necessary to understand the basics of how sounds work in the real world. It is safe to say that many people don’t ask this question – they just produce or react to them, unconsciously knowing the outcomes. Human mind has already developed a deep understanding of which sounds are produced under different circumstances – you can easily say which sounds to expect when somebody scratches a blackboard or rings a bell. Some could say that sounds are just “pressure waves that propagate through the air”, but in reality, there is a lot of interesting and complex things beyond this definition to pay attention to. This chapter will introduce the reader to the underlying physics of sound and different types of them. A special focus will be made on describing harmonic sounds, which are essential to understand to be able to work with music and pitch.\\

\section{What a Sound Is}

The definition of sound above, saying that it is just vibrations in the air, is hard to be called incorrect. Of course, there are improvements to be made: for example, that the sound can propagate not only through the air, but through any medium that has inert mass and is “elastic”, or stiff, meaning that it will respond to forces applied to it. It is also important to note that the definition above relates to sound as a physical phenomenon, but there is another definition used in psychology and physiology, saying that the sound is a perception in the brain, or auditory sensation of the concept described above. In this thesis, the term “sound” will be used primarily in the first (physical) sense, unless specified differently.\\

Now, the mass and elasticity of the air mentioned above play a very important role in the studies of sound. Mass-spring systems are a highly discussed topic, along with the type of oscillations they tend to have. Any object that can produce sounds can be considered a mass-spring system: a bell, a guitar string, or even air and water, which can be thought of as many small masses connected by invisible springs\dots{} If one imagines the simplest of such systems --- any mass attached to an elastic spring --- they can notice that when a particular force is applied to it, it tends to oscillate in a sinusoidal manner. In fact, this is true for all such systems: they naturally “want” to vibrate in a sinusoidal fashion with a preferred frequency, called resonance frequency.\\

The knowledge that any source of sound can be thought of as a mass-spring system is quite staggering. In most cases, it is hard to imagine such a system, because there could be no obvious mass nor elasticity, like in examples for resonant cavities: why a can of soda makes that clicking sound when it is being opened? The air is the answer. When you open the can, some parts of the air near its top act as a mass, and other parts near the bottom as a spring. The pressure in the can drops, and the “spring” at the bottom tries to suck the “mass” back in, producing the expected sound [Auditory Neuroscience citation].\\

Another essential topic to mention is why sounds fade in time. This is again connected to the concept of mass-spring systems and the amplitudes of their vibrations. Usually, the bigger these amplitudes are, the louder the resulting sound is, so if the amplitudes didn’t become smaller, we would live in constant unbearable noise. In brief, the fading is caused by the resistance of the medium, in which the sound propagates, and the manner of this propagating. If you imagine air as above --- as many masses connected by invisible springs --- the mechanics of the propagation becomes clear: the sound source pushes the closest mass near it, which due to elasticity pushes its neighbors and returns to its starting location. Then its neighbors, in turn, push their neighbors and return, and so on, until these vibrations come to your ears. The air masses must be pushed again and again for the sound to spread, so it tends to lose its strength along the way, and the further from its source it travels, the smaller the amplitudes of the vibrations become.\\

\section{Harmonicity and Pitch}


